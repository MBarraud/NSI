\chapter*{Avant-propos}


\vspace{\stretch{4}}
Dans le programme officiel :

\smallskip 

<< Le programme est organisé autour de huit rubriques. Il ne constitue cependant pas un plan de cours. Il appartient aux professeurs de choisir leur progression, sans faire de chaque partie un tout insécable et indépendant des autres. Au contraire, les mêmes notions peuvent être développées et éclairées dans différentes rubriques, en mettant en lumière leurs interactions. >>

\vspace{\stretch{2}}

L'objectif de ce document n'est pas de proposer une progression ni de faire de longues explications de cours, mais au contraire d'apporter une vue synthétique de chaque notion. Ainsi, et mise à part la dimension historique, trop transversale, cette présentation conserve l'organisation en parties du programme officiel. Il a fallu régulièrement faire des choix dans les développements, mais le principal était ici d'apporter le vocabulaire et les mots clés nécessaires à des recherches et des approfondissements personnels.

\vspace{\stretch{1}}
J'espère que ce document pourra être une ressource utile aux élèves qui y trouveront une sorte de lexique détaillé des connaissances à acquérir en première NSI. 

\vspace{\stretch{2}}

\begin{flushright}
Mickaël BARRAUD

mickael.barraud@ac-nantes.fr

Lycée Jean de Lattre De Tassigny

85000 La Roche-sur-Yon
\end{flushright}

\vspace{\stretch{8}}